\documentclass[aspectratio=169,notes]{beamer}
\usepackage{lmodern}
\usepackage{adjustbox}
\usepackage[T1]{fontenc}
\usepackage{textcomp}
\usepackage{animate}
\usepackage{underscore}
\usepackage{xmpmulti}
\usepackage{multimedia}
\usepackage{epstopdf}
\usepackage{soul}
\usepackage{bbding}
\usepackage[absolute,overlay]{textpos}
\usepackage[most]{tcolorbox}

\definecolor{Title}{rgb}{0.94,0.52,0.08}
\setbeamercolor{frametitle}{bg=Title,fg=black}

% footnote without number
\makeatletter
\def\blfootnote{\xdef\@thefnmark{}\@footnotetext}
\makeatother

\usepackage{hyperref}
\usepackage{scalerel}
\def\thumbup{\scalerel*{\includegraphics{thumbup.png}}{O}}
\usepackage{listings}
\lstdefinelanguage{D}
{
  % list of keywords
  morekeywords={ abstract, alias, align, asm, assert, auto, body, bool, break,
	byte, case, cast, catch, cdouble, cent, cfloat, char, class, const,
	continue, default, double, else, enum, export, extern, false, final, finally,
	float, for, foreach, foreach_reverse, function, goto, idouble, if, ifloat,
	immutable, import, in, inout, int, interface, invariant, ireal, is, lazy,
	long, mixin, module, new, nothrow, null, out, override, package, pragma,
	private, protected, public, pure, real, ref, return, scope, shared, short,
	static, string, struct, super, switch, synchronized, template, this, throw,
	true, try, typeid, typeof, ubyte, ucent, uint, ulong, union,
	unittest, delegate, @safe, @property
	ushort, version, void, wchar, while, with, __FILE__, __FILE_FULL_PATH__,
	__MODULE__, __LINE__, __FUNCTION__, __PRETTY_FUNCTION__, __gshared,
	__traits, __vector, __parameters
  },
  otherkeywords= { @property, @safe },
  sensitive=false, % keywords are not case-sensitive
  morecomment=[l]{//}, % l is for line comment
  morecomment=[s]{/*}{*/}, % s is for start and end delimiter
  morecomment=[s]{/+}{+/}, % s is for start and end delimiter
  morestring=[b]{"}, % defines that strings are enclosed in double quotes
  morestring=[b]{`} % defines that strings are enclosed in double quotes
}
\usepackage{color}
\definecolor{eclipseBlue}{RGB}{42,0.0,255}
\definecolor{eclipseGreen}{RGB}{63,127,95}
\definecolor{eclipsePurple}{RGB}{127,0,85}

\makeatletter

% here is a macro expanding to the name of the language
% (handy if you decide to change it further down the road)
\newcommand\language@yaml{yaml}

\expandafter\expandafter\expandafter\lstdefinelanguage
\expandafter{\language@yaml}
{
  keywords={true,false,null,y,n},
  keywordstyle=\color{darkgray}\bfseries,
  basicstyle=\YAMLkeystyle,                                 % assuming a key comes first
  sensitive=false,
  comment=[l]{\#},
  morecomment=[s]{/*}{*/},
  commentstyle=\color{purple}\ttfamily,
  stringstyle=\YAMLvaluestyle\ttfamily,
  moredelim=[l][\color{orange}]{\&},
  moredelim=[l][\color{magenta}]{*},
  moredelim=**[il][\YAMLcolonstyle{:}\YAMLvaluestyle]{:},   % switch to value style at :
  morestring=[b]',
  morestring=[b]",
  literate =    {---}{{\ProcessThreeDashes}}3
                {>}{{\textcolor{red}\textgreater}}1
                {|}{{\textcolor{red}\textbar}}1
                {\ -\ }{{\mdseries\ -\ }}3,
}

% switch to key style at EOL
\lst@AddToHook{EveryLine}{\ifx\lst@language\language@yaml\YAMLkeystyle\fi}
\makeatother

% Set Language
\lstset{
  language={D},
  basicstyle=\small\ttfamily, % Global Code Style
  captionpos=b, % Position of the Caption (t for top, b for bottom)
  extendedchars=true, % Allows 256 instead of 128 ASCII characters
  tabsize=2, % number of spaces indented when discovering a tab
  columns=fixed, % make all characters equal width
  keepspaces=true, % does not ignore spaces to fit width, convert tabs to spaces
  showstringspaces=false, % lets spaces in strings appear as real spaces
  breaklines=true, % wrap lines if they don't fit
  numbers=left, % show line numbers at the left
  numberstyle=\tiny\ttfamily, % style of the line numbers
  commentstyle=\color{eclipseGreen}, % style of comments
  keywordstyle=\color{eclipsePurple}, % style of keywords
  stringstyle=\color{eclipseBlue}, % style of strings
}
\definecolor{lightgray}{rgb}{.9,.9,.9}
\definecolor{darkgray}{rgb}{.4,.4,.4}
\definecolor{purple}{rgb}{0.65, 0.12, 0.82}
\lstdefinelanguage{TypeScript}{
	keywords={break, case, catch, continue, debugger, default, delete, do, else,
		false, from, finally, for, function, if, in, instanceof, new, null, return, switch,
		this, throw, true, try, typeof, var, void, while, with, interface,
		class, export, boolean, throw, implements, import, this, const, let,
		of, =>},
	morecomment=[l]{//},
	morecomment=[s]{/*}{*/},
	morestring=[b]',
	morestring=[b]",
	ndkeywords={},
	keywordstyle=\color{blue}\bfseries,
	ndkeywordstyle=\color{darkgray}\bfseries,
	identifierstyle=\color{black},
	commentstyle=\color{purple}\ttfamily,
	stringstyle=\color{red}\ttfamily,
	sensitive=true
}

\colorlet{punct}{red!60!black}
\definecolor{background}{HTML}{EEEEEE}
\definecolor{delim}{RGB}{20,105,176}
\colorlet{numb}{magenta!60!black}

\lstdefinelanguage{GraphQL}{
    basicstyle=\normalfont\ttfamily,
    numbers=left,
    stepnumber=1,
    showstringspaces=false,
    breaklines=true,
	keywords={type, schema, mutation, subscription, __type, __schema, kind,
		on, fragment, query},
    literate=
     *{0}{{{\color{numb}0}}}{1}
      {1}{{{\color{numb}1}}}{1}
      {2}{{{\color{numb}2}}}{1}
      {3}{{{\color{numb}3}}}{1}
      {4}{{{\color{numb}4}}}{1}
      {5}{{{\color{numb}5}}}{1}
      {6}{{{\color{numb}6}}}{1}
      {7}{{{\color{numb}7}}}{1}
      {8}{{{\color{numb}8}}}{1}
      {9}{{{\color{numb}9}}}{1}
      {:}{{{\color{punct}{:}}}}{1}
      {,}{{{\color{punct}{,}}}}{1}
      {\{}{{{\color{delim}{\{}}}}{1}
      {\}}{{{\color{delim}{\}}}}}{1}
      {[}{{{\color{delim}{[}}}}{1}
      {]}{{{\color{delim}{]}}}}{1},
}

\lstdefinelanguage{json}{
    basicstyle=\normalfont\ttfamily,
    numbers=left,
    stepnumber=1,
    showstringspaces=false,
    breaklines=true,
    literate=
     *{0}{{{\color{numb}0}}}{1}
      {1}{{{\color{numb}1}}}{1}
      {2}{{{\color{numb}2}}}{1}
      {3}{{{\color{numb}3}}}{1}
      {4}{{{\color{numb}4}}}{1}
      {5}{{{\color{numb}5}}}{1}
      {6}{{{\color{numb}6}}}{1}
      {7}{{{\color{numb}7}}}{1}
      {8}{{{\color{numb}8}}}{1}
      {9}{{{\color{numb}9}}}{1}
      {:}{{{\color{punct}{:}}}}{1}
      {,}{{{\color{punct}{,}}}}{1}
      {\{}{{{\color{delim}{\{}}}}{1}
      {\}}{{{\color{delim}{\}}}}}{1}
      {[}{{{\color{delim}{[}}}}{1}
      {]}{{{\color{delim}{]}}}}{1},
}
\usepackage{tikz}
\usetikzlibrary{shadows,calc}
\usepackage{xkeyval}
\usepackage{todonotes}
\presetkeys{todonotes}{inline}{}
\defbeamertemplate{description item}{align left}{\insertdescriptionitem\hfill}
\usetheme{metropolis}					 % Use metropolis theme
\usepackage[
    backend=biber,
	sorting=none,
    url=true
]{biblatex}
\addbibresource{biblio.bib}
\setbeamertemplate{bibliography item}{\insertbiblabel}

\usefonttheme{professionalfonts}
\usefonttheme{serif}
\usepackage{fontspec}
\setmainfont{Nimbus Sans}

\title{Good Fun: Creating a Data-Oriented Parser/AST/Visitor Generator}
\date{DConf 2024}
\author{Dr.\,Robert Schadek}

\begin{document}
	\maketitle

	\begin{frame}[fragile]{Why}
		\begin{itemize}
			\item I like to work on parser generators
			\pause
			\begin{itemize}
				\item I do not need them
				\item I do not like to use them for something useful
			\end{itemize}
			\item but they are really good fund
		\end{itemize}
	\end{frame}

	\begin{frame}[fragile]{A bit of darser history}
		darser is a parser generator that
		\begin{itemize}
			\item generates a recursive decent parser
			\item generates AST classes for parser to use
			\item generates visitor to traverse
			\item is used in graphqld
		\end{itemize}
	\end{frame}

	\begin{frame}[fragile]{Input}
		\lstinputlisting[language=D,firstline=1,lastline=4,basicstyle=\scriptsize\ttfamily]{example.d}
	\end{frame}

	\begin{frame}[fragile]{AST 1/3}
		\lstinputlisting[language=D,firstline=6,lastline=12,basicstyle=\scriptsize\ttfamily]{example.d}
	\end{frame}

	\begin{frame}[fragile]{AST 2/3}
		\lstinputlisting[language=D,firstline=14,lastline=27,basicstyle=\scriptsize\ttfamily]{example.d}
	\end{frame}

	\begin{frame}[fragile]{AST 3/3}
		\lstinputlisting[language=D,firstline=29,lastline=44,basicstyle=\scriptsize\ttfamily]{example.d}
	\end{frame}

	\begin{frame}[fragile]{Parser Example 1 1/2}
		\lstinputlisting[language=D,firstline=46,lastline=65,basicstyle=\scriptsize\ttfamily]{example.d}
	\end{frame}

	\begin{frame}[fragile]{Parser Example 1 2/2}
		\lstinputlisting[language=D,firstline=71,lastline=75,basicstyle=\scriptsize\ttfamily]{example.d}
	\end{frame}

	\begin{frame}[fragile]{Visitor}
		\lstinputlisting[language=D,firstline=77,lastline=96,basicstyle=\scriptsize\ttfamily]{example.d}
	\end{frame}

	\begin{frame}[fragile]{Input 2}
		\lstinputlisting[language=D,firstline=98,lastline=102,basicstyle=\scriptsize\ttfamily]{example.d}
	\end{frame}

	\begin{frame}[fragile]{Parser Example 2 1/2}
		\lstinputlisting[language=D,firstline=104,lastline=122,basicstyle=\scriptsize\ttfamily]{example.d}
	\end{frame}

	\begin{frame}[fragile]{Parser Example 2 2/2}
		\lstinputlisting[language=D,firstline=123,lastline=139,basicstyle=\scriptsize\ttfamily]{example.d}
	\end{frame}

	\section{Data-oriented Design (DoD)}

	\begin{frame}[fragile]{Data-oriented Design (DoD)}
		\large
		\centering Putting data that is accessed together in arrays, while
			making sure that every bit counts!
	\end{frame}

	\begin{frame}[fragile]{Hardware}
		\centering
		\begin{tabular}{r r r}
				& time & size \\ \hline
		   INST & $\approx{}$ 0.25-10 cycles & 128B \\
\pause
			L1 & 3 cycles & 16KB - 128 KB \\
			L2 & 10 cycles & 256KB - 1MB \\
			L3 & 40 cycles & 2MB - 32MB \\
		   RAM & 100 cycles & how much money do you have
		\end{tabular}

	\end{frame}

	\begin{frame}[fragile]{Why use Arrays}
		\begin{itemize}
			\item L1 cache lines are loaded one line at a time
			\item chances are good that after reading one array element you read the next
			\item Pointers on 64bit system are wasteful
			\pause
			\item At least on current 64bit Linux you can only address $2^{48}$ bit.
			\vspace{1cm}
			\item If an \lstinline@uint@ index is not good enough, reconsider
your decisions
		\end{itemize}
	\end{frame}

	\begin{frame}[fragile]{What now}
		\begin{itemize}
			\item Its called Abstract Syntax Tree not Abstract Syntax Array
			\pause
			\vspace{1cm}
			\item But what is an Tree with Nodes and Pointers then indices into the
ultimate array that is main memory.
			\pause
			\vspace{1cm}
			\item How hard can it be
		\end{itemize}
	\end{frame}

	\begin{frame}[fragile]{AST Array}
		\lstinputlisting[language=D,firstline=142,lastline=148,basicstyle=\scriptsize\ttfamily]{example.d}
	\end{frame}

	\begin{frame}[fragile]{Parser Array 1/3}
		\lstinputlisting[language=D,firstline=150,lastline=165,basicstyle=\scriptsize\ttfamily]{example.d}
	\end{frame}

	\begin{frame}[fragile]{Parser Array 2/3}
		\lstinputlisting[language=D,firstline=178,lastline=192,basicstyle=\scriptsize\ttfamily]{example.d}
	\end{frame}

	\begin{frame}[fragile]{Parser Array 3/3}
		\lstinputlisting[language=D,firstline=192,lastline=201,basicstyle=\scriptsize\ttfamily]{example.d}
	\end{frame}

	\begin{frame}[fragile]{Visitor Array 1/3}
		\lstinputlisting[language=D,firstline=203,lastline=218,basicstyle=\scriptsize\ttfamily]{example.d}
	\end{frame}

	\begin{frame}[fragile]{Results}
		\centering
		\begin{tabular}{l r r}
			Measure & class based & struct based \\ \hline
			Wall Clock & 8.6s & 7.1s \\
			L1-dcache-loads & $15\_884\_421\_092$ & $10\_671\_708\_742$ \\
			L1-dcache-load-misses & $258\_055\_781$ & $200\_917\_391$ \\
			L1-misses-percentage & $1.62\%$ & $1.88\%$ \\
			Maximum resident set size & $585\_984 KiB$ & $193\_668 KiB$
		\end{tabular}
	\end{frame}

	\section{Structured Ranting}

	\begin{frame}[fragile]{AST Re-Structuring}
		\lstinputlisting[language=D,firstline=257,lastline=266,basicstyle=\scriptsize\ttfamily]{example.d}
	\end{frame}

	\begin{frame}[fragile]{AST Re-Structuring}
		\lstinputlisting[language=D,firstline=268,lastline=279,basicstyle=\scriptsize\ttfamily]{example.d}
	\end{frame}

	\begin{frame}[fragile]{Reading/Writing AST on Disk 1/2}
		\lstinputlisting[language=D,firstline=296,lastline=304,basicstyle=\scriptsize\ttfamily]{example.d}
	\end{frame}

	\begin{frame}[fragile]{Reading/Writing AST on Disk 2/2}
		\lstinputlisting[language=D,firstline=306,lastline=318,basicstyle=\scriptsize\ttfamily]{example.d}
	\end{frame}

	\begin{frame}[fragile]{Lexer and Tokens}
		\begin{center}
		\large
		Lexers and Tokens are no fun \pause so much manual work
		\end{center}
	\end{frame}

	\begin{frame}[fragile]{TokenType}
		\lstinputlisting[language=D,firstline=321,lastline=334,basicstyle=\scriptsize\ttfamily]{example.d}
	\end{frame}

	\begin{frame}[fragile]{Lexer and Tokens}
		\lstinputlisting[language=graphql,firstline=281,lastline=293,basicstyle=\scriptsize\ttfamily]{example.d}

%\$activeAfter, \$includedInHeadcount, \$knownAsName, \$legalName, \$privateContact, activeAfter, Boolean, createPerson, DateTime, id, includedInHeadcount, knownAsName, KnownAsNameIn, legalName:, LegalNameIn, MutateCreatePerson, mutation, privateContact, PrivateContactIn
		That graphql only contains 20 strings that need storing
	\end{frame}
\end{document}
